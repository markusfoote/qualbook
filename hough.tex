In image processing the \textbf{Hough Transform} is useful for \textit{global} filtering, especially to find sets of pixels that lie along curves of a specified shape. It provides an elegant solution using a \textit{discretized parameter space}, while the na\"ive, brute-force approach quickly becomes daunting. In the case of finding lines in an image of $n$ points, the na\"ive approach involves iteration over all $\sim\!\! n^2$ possible lines and performing $\sim\!\! n^3$ comparisons for every point to each line. The $O(n^3)$  complexity exponentially worsens for shapes with higher dimensional parameter spaces. This approach is computationally prohibitive for non-trivial applications.

Instead, the Hough Transform accumulates non-background points in a discretized parameter space. The dimensionality of the parameter space is equal to the number of the parameters that describe the desired shape. In the case of a line, there are two parameters. However, the parameters must be defined with care. The na\"ive parameter choices for a line might be \textit{slope} and \textit{intercept}, but if the objective is to find vertical (or nearly vertical) lines, a method to accurately discretize infinity (or very large values) would be required. A better approach is to parameterize lines by their \textit{normal} representation:
\begin{equation}
x \cos (\theta) + y \sin(\theta) = \rho \; .
\end{equation}
In this parameterization, all parameters are bounded: $-\sfrac{\pi}{2} \leq \theta \leq \sfrac{\pi}{2}$ and $-D\leq \rho \leq D$ for an image with D length between the two most distant corners.