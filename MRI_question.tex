\section{Overview} %TODO Maybe remove this
\textbf{Magnetic Resonance Imaging} (\textbf{MRI}) is an imaging modality that expoits the phenomenon of Nuclear Magnetic Resonance to record information about the local chemical environment of specific atoms within a body. Commonly, these atoms are the hydrogen nuclei, but MRI is possible with any nuclei with a non-integer nuclear spin. Generally, the proceedure for MRI is summarized in the following steps, where only step \ref{item:mri:gensteps:gradients} is specific to the Imaging modality over an NMR experiment:
\begin{enumerate}
	\item \label{item:mri:gensteps:b0} Nuclei align with an applied, strong magnetic field $\vec{B}_0$.
	\item \label{item:mri:gensteps:rf} Radio Frequency energy is applied to flip the nuclear magnetization off-axis from $\vec{B}_0$.
	\item \label{item:mri:gensteps:precess} Interaction between $\vec{B}_0$ and nuclear magnetizations causes magnetizations to \textbf{\textit{precess}}.
	\item \label{item:mri:gensteps:record} Induced current from precessing magnetization is observed in an RF coil.
	\item \label{item:mri:gensteps:gradients} Speed of precession is spatially varied by a spatially-varying magnetic field gradient.
\end{enumerate}
\section{Magnetism and Nuclear Magnetic Resonance}
Magnetism is an inherent property of matter which causes materials to interact with external magnetic fields to generate their own magnetic field. This phenomenon is characterized by the relation
\begin{equation}
\vec{H} = \chi \vec{B} \label{eq:mri:magneticsusceptibility}
\end{equation}
where $\vec{H}$ is the generated field, $\vec{B}$ is the applied field, and $\chi$ is the \textit{magnetic susceptibility} of the matter. Matter is classified into three main categories based on $\chi$:
\begin{description}
	\item[Paramagnetic] matter has \textit{positive} $\chi$, thus $\vec{H}$ points in the same direction as $\vec{B}$. \textbf{Ex:} Aluminum
	\item[Diamagnetic] matter has \textit{negative} $\chi$, thus $\vec{H}$ points in the opposite direction as $\vec{B}$. \textbf{Ex:} Water
	\item[Ferromagnetic] matter has \textit{large} $\chi$. \textbf{Ex:} Iron
\end{description}
Magnetism originates from the electrons of an atom, but Nuclear Magnetic Resonance originates from the nucleus and is still subject to electronic magnetism, or \textit{chemical shift} phenomena.

Nuclear Magnetism arises from nuclei of atoms with odd-even pairing of neutrons and protons, resulting in nuclei with non-integer spin. E.g. $^1$H with spin $I=\sfrac{1}{2}$. Nuclear spin in motion gives rise to an angular momentum of the spin:
\begin{equation}
J=\hbar I = \frac{h}{2\pi}I
\end{equation}
where J is the resulting angular momenta, $h$ is Planck's constant, and $\hbar$ is the reduced Planck constant. 
The magnetic moment of matter is also determined by
\begin{equation}
\mu = \gamma J = \gamma \hbar I
\end{equation}
where $\mu$ is the magnetic moment and $\gamma$ is the \textit{gyromagnetic ratio} that is specific to the nucleus in question.

Bulk magnetization (the collective net magnetization from a region of matter) is dependent upon the relative number of individual moments in the 'up' (parallel to $\vec{B}_0$) and 'down' (anti-parallel to $\vec{B}_0$) states:
\begin{equation}
\vec{m} = \mu (\Delta N)
\end{equation}
where $N$ is the number of magnetic moments in the up and down states such that $\Delta N$ is the net difference of moments in the anti-parallel direction.
$N$ follows a Boltzmann distribution based on the energy state, $E$, of the nuclei:
\begin{equation}
N\propto e^{-\frac{E}{kT}}
\end{equation}
where $k$ is the Boltzman constant and $T$ is the absolute temperature. The energy state of a magnetic moment is
\begin{equation}
E(\mu) = -\mu B_0 = -\gamma \hbar m B_0
\end{equation}
where $m = \left\{I \ldots\sfrac{3}{2}, \sfrac{1}{2}, -\sfrac{1}{2}, -\sfrac{3}{2},\ldots -I\right\}$ are the permissible states for the spin $I$. Again assuming $^1$H nuclei, the two energy states are
\begin{align}
E\left(\text{up}\right) &= - \gamma \hbar \left(\sfrac{1}{2}\right) B_0 \\
E\left(\text{down}\right) &= + \gamma \hbar \left(\sfrac{1}{2}\right) B_0 .
\end{align}
The ratio of spin alignment can then be calculated:
\begin{equation}
\frac{N\left(\text{up}\right)}{N\left(\text{down}\right)} = e^{+ \frac{\Delta E}{kT}} \label{eq:mri:ratio}
\end{equation}
where $\Delta E = E\left(\text{down}\right) - E\left(\text{up}\right)$. If we then suppose that there exist a total of $N$ nuclei such that $N\left(\text{up}\right) + N\left(\text{down}\right) = N$, then from this and (\ref{eq:mri:ratio}) we can solve for $\Delta N$ and thus $M$:
\begin{equation}
M=\mu \Delta N = \frac{N\gamma^2 \hbar^2 B_0}{4kT}
\end{equation}
This equation can be massaged to match the form of (\ref{eq:mri:magneticsusceptibility}):
\begin{align}
M = \chi_{_0} B_0 && \chi_{_0} = \frac{N\gamma^2 \hbar^2 }{4kT}
\end{align}
where $\chi_{_0}$ is a matter-specific term that describes the resulting magnetic field from an applied field $B_0$. This $\chi_{_0}$ is very small, and results in a very low effective signal from a body of matter. However, there are engineering decisions in each of these terms that helps raise the signal:
\begin{description}
	\item[$\mathbf{B_0}$] Make it big, but not too big. Like 1.5 T at least, probably 7 T is awesome.
	\item[$\mathbf{N}$] Effectively this is the voxel size. Cubic power. $\sfrac{1}{2}$ resolution $\to$ 8x larger voxel $\to$ 64x signal.
	\item[$\mathbf{T}$] Very limited range if you want the subject to be alive, but colder is better.
	\item{$\!\!\mathbf{\gamma}$} Pick nuclei with high $\gamma$, and isotope with high abundance. See Table \ref{tab:mri:nuclist}.
\end{description}
\begin{table}[h]
	\caption{\label{tab:mri:nuclist} Relevant Properties of Some Common Nuclei}
	\centering
	\begin{tabular}{|c|c|c|c|}
		\hline Nuclei & $I$ & $\frac{\gamma}{2\pi}$ (MHz/T) & nat. abundance (\%) \\ 
		\hline $^1$H & \sfrac{1}{2} & \textbf{42.58} & 99.98 \\ 
		\hline $^{13}$C & \sfrac{1}{2} & 10.71 & 1.11 \\ 
		\hline $^{19}$F & \sfrac{1}{2} & 40.05 & 100 \\ 
		\hline $^{31}$P & \sfrac{1}{2} & 17.23 & 100 \\ 
		\hline $^{23}$Na & \sfrac{3}{2} & 11.26 & 100 \\ 
		\hline 
	\end{tabular} 
\end{table}

\section{Spin Motion and the NMR Signal}


\section{Spatial Encoding}
