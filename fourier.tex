The {\em Fourier transform} is analogous to the separation of white light into a rainbow by a prism. This mathematical operator has many useful applications in imaging. We define the Fourier transform of a function $g$ as
\begin{equation}
G(f)=\mathcal{F}\left\{g(t)\right\} = \int_{-\infty}^{\infty} g(t) e^{-i2\pi ft}\, dt \;.
\end{equation}
We define the inverse of this operation as
\begin{equation}
g(t)=\mathcal{F}^{-1}\left\{G(f)\right\} = \frac{1}{2\pi}\int_{-\infty}^{\infty} G(f) e^{i2\pi ft}\, df \;.
\end{equation}
These definitions are not unique, especially in the handling of the normalizing factors; it is also common to see a $\sfrac{1}{\sqrt{2\pi}}$ coefficient in \emph{both} forward and inverse operations.

The Fourier transform can also be considered as a linear operation (which it is), in terms of vectors and matrices.

\todo{examples here}

The Fourier transform has many useful properties, some of which are listed in Table \ref{tab:fourierProperties} .

\begin{table}
	\caption{\label{tab:fourierProperties} Fourier Transform Properties}
	{\renewcommand{\arraystretch}{2}%
	\begin{tabular}{|c|M{6.5cm}|p{4.5cm}|}
		\hline
		{\textbf{Name}} & {\textbf{Formula}} &{\textbf{Notes}} \\
		\hline\hline
		Linearity& $\mathcal{F}\left\{\alpha\, g_1(t) + \beta\, g_2(t)\right\} = \alpha\, G_1(f) + \beta\, G_2(f)$ & $\alpha$, $\beta$ may be complex.\\
		\hline
		Scaling & $\mathcal{F}\left\{g(\alpha t) \right\} = \frac{1}{|\alpha|}G(\frac{f}{\alpha})$& $\alpha > 1$ compresses $g$ in time $\to$ expands in frequency\\
		\hline
		Duality & $\mathcal{F}\left\{G(t)\right\}=g(-f)$& \\
		\hline
		Shifting & $\mathcal{F}\left\{g(t-\tau)\right\}= G(f) e^{-i2\pi f\tau}$  & \\
		\hline
		Convolution & 
		{$\begin{aligned}%
		\mathcal{F}\left\{ g_1(t) \circledast g_2(t)\right\} & =  G_1(f) \cdot G_2(f) \\ \mathcal{F}\left\{ g_1(t) \cdot g_2(t)\right\} & = G_1(f) \circledast G_2(f)
		\end{aligned}$}&\\
		\hline
	\end{tabular}}
	
\end{table}

