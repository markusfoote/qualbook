The {\em Fourier transform} is analogous to the separation of white light into a rainbow by a prism. This mathematical operator has many useful applications in imaging. We define the Fourier transform of a function $g$ as
\begin{equation}
G(f)=\mathcal{F}\left\{g(t)\right\} = \int_{-\infty}^{\infty} g(t) e^{-i2\pi ft}\, dt \;.
\end{equation}
We define the inverse of this operation as
\begin{equation}
g(t)=\mathcal{F}^{-1}\left\{G(f)\right\} = \frac{1}{2\pi}\int_{-\infty}^{\infty} G(f) e^{i2\pi ft}\, df \;.
\end{equation}
These definitions are not unique, especially in the handling of the normalizing factors; it is also common to see a $\sfrac{1}{\sqrt{2\pi}}$ coefficient in \emph{both} forward and inverse operations.

The Fourier transform can also be considered as a linear operation (which it is), in terms of vectors and matrices.
%TODO example here

The Fourier transform has many useful properties, some of which are listed in Table \ref{tab:fourierProperties} .

\begin{table}
	\caption{\label{tab:fourierProperties} Fourier Transform Properties}
	{\renewcommand{\arraystretch}{2}%
	\begin{tabular}{|c|c|l|}
		\hline
		Linearity& $\mathcal{F}\left\{\alpha\, g_1(t) + \beta\, g_2(t)\right\} = \alpha\, G_1(f) + \beta\, G_2(f)$ & Note: $\alpha$, $\beta$ may be complex.\\
		\hline
		Scaling & mote & and more\\
		\hline
		Duality && \\
		\hline
		Shifting &  &\\
		\hline
		Convolution &  &\\
		\hline
	\end{tabular}}
	
\end{table}

